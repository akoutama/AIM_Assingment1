\documentclass{article}
\usepackage[utf8]{inputenc}

\title{Assingment1}
\author{Maria Akouta}
\date{May 2019}

\begin{document}

\maketitle

\section*{Task 1 - XML Schema}
\subsection*{1.1}
XML well-formated rules:
\begin{itemize}
    \item A well-formed XML document has a single top-level element, meaning that if every tag in the document was collapsed, only the top-level element would be visible.
    \item Is it properly nested, meaning every tag in the document has  matching start and end tags, which are in the same parent element. This means an ending tag to a starting tag may not break out of the enclosing tag of the startung tag
\end{itemize}{}
The given XML document ist well-formed, as it has a single top-level element and all tags are properly nested in their parent contexts.

\subsection*{1.2}
The document is well-formed but invalid.
\begin{table}[h]
\begin{tabular}{|l|l|}
\hline
Line number & Invalidity \\ \hline
19          & \begin{tabular}[c]{@{}l@{}}the StreetNumber element is missing in the adress. \\ It is required in the schema in line 61\end{tabular} \\ \hline
26          & according to line 42 in the schema, an item can only have a maximum of value 9 \\ \hline
41          & the attribute adressID must be present, as required by line 71 in the schema \\ \hline
44          & the ZipCode element needs to have 5 digits, according to line 65 in the schema \\ \hline
48 and 53   & there are multiple occurences of the same value for the itemID attribute          \\ \hline
\end{tabular}
\end{table}
\subsection*{1.3}
No.\\
A valid XML document must be well-formed.


Validity implies well-formed, but well-formed doesn't imply validity.
\end{document}
